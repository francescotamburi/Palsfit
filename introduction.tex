\chapter{Intro}

\section{Positron Annihilation Lifetime Spectroscopy}

Positron annihilation as a means of investigating lattice imperfections traces its roots back to the late 1960s, when the link between positron lifetimes and lattice vacancies was first proposed. \cite{Mackenzie67} As the lifetime of a positron is a function of the electron density at the annihilation site, and lattice defects (e.g. open vacancies) represent a local decrease in electron density, longer lifetimes are observed for positrons trapped in these defects. \cite{Krause} The presence of multiple positron lifetimes within a sample material, thus, indicates the presence of lattice defects within that material.

Positron Annihilation Lifetime Spectroscopy (PALS) is the study of these lifetimes, with the aim of determining the number and types of defects present in a sample material. While other techniques to study lattice defects exist, the main advantage of PALS is the high sensitivity of the technique to open-volume defects.\footnote{Thorough comparisons of defect sensitive techniques, as well as their respective advantages and disadvantages, are available in literature, but detailed analysis of these is beyond the scope of this project.} This property, along with the non-destructive nature of the technique, has led to its application in various fields of study.\cite{PharmApp} \cite{Nuc} \todo{rewrite this sentence, highlight nuclear and drop med ref, if anything}

\subsection{The Positron Lifetime}

Weakly radioactive isotopes, such as \ch{^{22}Na}, are commonly used to generate positrons. As part of its {$\beta^{+}$} decay, \ch{^{22}Na} simultaneously produces a positron and a 1.27 MeV $\gamma$ photon. Upon encountering the sample material, positrons quickly lose their kinetic energy in a process known as "thermalization".

The thermalization process isn't uniform, occurring via different mechanisms depending on the sample material and energy of the positron. At high energies, for example, atomic ionization is a dominant process, while at low energies ($<1$ eV in metals and $\sim1$ eV in semiconductors), phonon scattering is the most important process. Regardless of specific mechanism, however, thermalization occurs within the first few picoseconds of a positron lifetime.

After thermalization, the positron is free to diffuse through the sample material. During the diffusion process, the wavefunction of the positively charged positron spreads out into a highly delocalized Bloch state, primarily concentrated in the interstitial space between atomic nuclei in the material. It's in this phase that the positron spends the bulk of its lifetime and where defect trapping predominantly occurs.\footnote{Some trapping can occur during the thermalization process. The effect of this on positron lifetimes, however, has been proven to be negligible.} How far a positron diffuses determines how many atoms are probed for positron traps, and thus how sensitive the PALS process is to defects.

Trapping occurs when a defect is encountered and a localized positron state forms at the defect. The rate at which this occurs depends on trapping rate $\kappa$, and is proportional to the defect concentration $C$. Specifically
\begin{equation}
    \kappa = \mu C,
    \label{eq:traprate}
\end{equation}
where $\mu$ is termed the positron trapping coefficient, and is constant for a given defect.

Annihilation can occur in the delocalized state or the localized trapped state, producing two 511 keV $\gamma$-photons. In both cases, lifetimes are usually on the order of hundreds of picoseconds.

\subsection{Positronium}

Positrons can also enter a third state, however, where, instead of delocalizing into the bulk or being trapped in a defect, the positron enters a bound state with an electron, forming an exotic atom known as positronium (Ps). Annihilation lifetimes for positronium depend on the spin alignment between the two particles and are around 140ps for the less common, parallel spin, ortho-Ps, or 1-5ns for the more common, antiparallel spin, para-Ps.

\subsection{Measuring Lifetimes}

Positron lifetimes can be determined experimentally by measuring the time interval between the emission of gamma photons that signal the production and annihilation of the positron. The experimental setup for doing so begins with a coupled scintillator and photomultiplier, which converts the $\gamma$-rays into analog electrical pulses. These are then processed by discriminators, distinguishing the 1.27 MeV (in the case of \ch{^{22}Na}) from the 511keV pulses. The former is used as a "start" signal to begin charging a capacitor, and the latter stops the charging process. The capacitor acts as a sort of "electronic stopwatch", converting the time delay into an amplitude signal. To ensure a linear time-amplitude relationship, the stop signal is delayed by a fixed amount. Each signal is stored in the memory of a multichannel analyzer, where the channel numbers represent time and every count represents a single annihilation event. The resulting lifetime spectrum contains more than $10^6$ annihilation events.

\subsection{The Spectrum}

A lifetime spectrum contains $k+1$ lifetime components, corresponding to $k$ different defect types and the positron lifetime in the defect-free bulk ($\tau_b$). From these we get the time-dependent positron decay spectrum $D(t)$ -- the probability for a positron to still be alive at a given time $t$ \cite{EFermi} -- given by the expression
\begin{equation}
    D(t) = \sum_{i=1}^{k+1}I_i\exp\left(-\frac{t}{\tau_i}\right),
    \label{eq:decayspec}
\end{equation}
where $\tau_i$ and $I_i$ represent the lifetime and intensity of each component, respectively.\footnote{When $k=0$ (i.e. there are no defect components), $\tau_i = \tau_b$ (the bulk lifetime) and $I_i = 100\%$} The resulting positron lifetime spectrum $N(t)$ is given by the absolute value of the time derivative of the positron decay spectrum $D(t)$\footnote{Due to the delay introduced in the time-amplitude conversion, $t = t + t_0$}:
\begin{equation}
    N(t) = \left|\frac{\mathrm{d}D(t)}{\mathrm{d}t}\right| = \sum_{i=1}^{k+1}\frac{I_i}{\tau_i}\exp\left(-\frac{t + t_0}{\tau_i}\right).
    \label{eq:lifespec}
\end{equation}

The experimental spectrum is the convolution of the lifetime spectrum analytically described above and an Instrument timing Resolution Function (IRF). This resolution function is mainly determined by the scintillator+photomultiplier and takes the form of a sum of several Gaussian functions, 
\begin{equation}
    F(t) = \sum G_i(t),
    \label{eq:IRF}
\end{equation}
with variations in the peak position, relative height and width of each Gaussian, $G_i(t)$. When the individual Gaussians are close enough to each other, $F(t)$ can be approximated by a single Gaussian. 

The experimental spectrum thus takes the form
\begin{equation}
    D_{exp} (t) = \int_{-\infty}^{+\infty} D(t-t') F(t') dt'.
\end{equation}
In addition, experimental spectra also contain a level of background noise and contributions from positron annihilation within the source.

\section{The Standard Trapping Model}

The goal of PALS is to extract meaningful information relating to defects in a sample material. The Standard Trapping Model (STM) links positron lifetimes to the number of defect types and their concentration. In the STM, the spectrum is modeled using a series of differential equations, based on annihilation and trapping rates, called rate equations. Two key assumptions of the model are that

\begin{enumerate}[label=(\roman*)]
    \item Positron trapping during thermalization can be safely ignored,
    \item Defects within the sample are homogeneously distributed.
\end{enumerate}

\subsection{Single Defect Trapping}

The simplest trapping model assumes a single, open volume defect type, such as an atomic vacancy. After thermalization, positrons either annihilate in the defect free bulk at a rate $\lambda_b = 1/\tau_b$ or are trapped in an open-volume defect, with a trapping rate $\kappa_d \propto$ the defect concentration $C$. As the electron density within the defect is lower than that of the bulk, the defect annihilation rate $\lambda_d = 1/\tau_d$ must be smaller than $\lambda_b$.

This information is used to write the rate equations:
\begin{equation}
    \frac{\mathrm{d}n_b(t)}{\mathrm{d}t} = -(\lambda_b + \kappa_d) n_b(t),
    \label{eq:ratebulk}
\end{equation}
\begin{equation}
    \frac{\mathrm{d}n_d(t)}{\mathrm{d}t} = - \kappa_d n_d(t) + \kappa_d n_b(t),
    \label{eq:ratedefect}
\end{equation}
which describes the change in the number of positrons in the bulk ($n_b$) and the defect ($n_d$), repectively, at time $t$. As we have assumed no trapping during thermalization, if we define $N_0$ as the number of positrons at $t=0$, we get our starting conditions, i.e. $n_b(0) = N_0$ and $n_d(0) = 0$. The solution to eq.s \ref{eq:ratebulk} and \ref{eq:ratedefect} gives us the decay spectrum of the positrons
\begin{equation}
    D(t) = \frac{\lambda_b - \lambda_d}{\lambda_b - \lambda_d + \kappa_d} e^{- (\lambda_b+\kappa_d) t} + 
    \frac{\kappa_d}{\lambda_b - \lambda_d + \kappa_d} e^{-\lambda_d t}.
\end{equation}

Taking eq. \ref{eq:decayspec}, and setting $k=1$ (as we're dealing with a single defect type) we can make the appropriate subtitutions and get:
\begin{equation}
    D(t) = I_1\exp\left(-\frac{t}{\tau_1}\right) + I_2 \exp \left(-\frac{t}{\tau_2}\right),
\end{equation}
where
\begin{equation}
    \begin{array}{lcl}
        \tau_1 = \dfrac{1}{\lambda_b+\kappa_d} &,& \tau_2 = \dfrac{1}{\lambda_d}, \\
        I_1 = 1 - I_2                          &,& I_2 = \dfrac{\kappa_d}{\lambda_b - \lambda_d + \kappa_d}.
    \end{array}
    \label{eq:tau1tau2}
\end{equation}

By taking the absolute value of the time derivative of the decay spectrum, in similar fashion to eq. \ref{eq:lifespec}, we get the positron lifetime spectrum
\begin{equation}
    N(t) = \left|\frac{\mathrm{d}D}{\mathrm{d}t}\right| = 
    \frac{I_1}{\tau_1} \exp \left(-\frac{t}{\tau_1}\right) +
    \frac{I_2}{\tau_2} \exp \left(-\frac{t}{\tau_2}\right).
    \label{eq:stmlifespec}
\end{equation}

\subsection{Observations}

Looking closely at eq. \ref{eq:stmlifespec}, we can see that there are two lifetime components in the spectrum. From eq. \ref{eq:tau1tau2} we can observe that the trapping rate $\kappa_d$ affects the first lifetime component and the relative intensity of the two components, but not the second lifetime component, which arises from positron annihilation within the defects. 

This means that while $\tau_2$, also called the defect-related lifetime $\tau_d = 1/\lambda_d$, is simply the positron lifetime within the defect, $\tau_1$ is not the positron lifetime within the defect-free bulk $\tau_b = 1/\lambda_b$. As $\tau_1 \leq \tau_b$, $\tau_1$ is also called the "reduced bulk" lifetime and as we increase the defect concentration (and thus increase the trapping rate $\kappa_d$) $\tau_{1}$ and its associated intensity $I_{1} \to 0$.

This limiting case, known as saturation trapping, happens when the defect concentration is high enough that the average spacing between defects is much smaller than the positron diffusion length. As a result, virtually all positrons are trapped, and the reduced bulk component disappears entirely from the spectrum.

\subsection{Multiple defects}

The single defect model can be easily extended in order to model more complex spectra, where more than one defect type is present. This is simply done by assuming no interaction between different defects and writing additional rate equations for each additional defect type. Thus, instead of having two rate equations, we have $k+1$ equations, with $k$ = the number of defects, or
\begin{equation}
    \dfrac{\mathrm{d}n_b}{\mathrm{d}t} = -\left(\lambda_b + \sum_{i=1}^{k} \kappa_i\right)n_b(t)
\end{equation}
for the bulk and
\begin{equation}
    \dfrac{\mathrm{d}n_{di}}{\mathrm{d}t} = \kappa_i n_b(t) - \kappa_i n_i(t)
\end{equation}
for each defect type, with $i = 1,\dots,k$.

\section{Spectrum Decomposition and Project Aims}

Decomposition of the lifetime spectrum is an example of an "inverse problem", where observations (the spectrum) are used to determine the causal factors that produced them (the components that make up the spectrum). The complexity of these problems means there are limits on when and how well they can be solved.

Lifetime decomposition is commonly done by using some sort of least-squares method to fit a model function to an experimental spectrum. One disadvantage of this approach is that the number of lifetime components in the model function must be estimated before performing the fit. Working around this limitation often means, then, starting with a single component fit and increasing the number of components until the variance of the fit stops decreasing.

The aim of this project is to investigate the fitting process at its limits, observing where the process fails and how it fails.

