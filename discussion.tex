\chapter{Conclusion}

A total of five trials were performed with PALS spectra that were simulated and fitted using PALSsim and PALSfit. The first two looked at how the interaction between lifetime components and intensities of two-lifetime spectra affected the fitting process. The second two gauged to what extent changing two experimental parameters, the width of the instrument timing resolution function and the number of counts, might improve fits. The third studied how the addition of a third component, corresponding to the effect of para-Ps formation, and the choice in number of components affected the fit.

\section{Findings}

A number of conclusions may be drawn from the first two trials. As expected, lifetime separation and relative intensity are both significant factors in how well two components can be resolved, and the effect of these can be seen in both the accuracy and precision of the fit. Larger lifetime separation leads to an improvement in fitted values, while higher intensity of a given component positively affects fitting results for that component's associated lifetime value, to the detriment of the values for other components. The value of the first lifetime component in relation to its separation also seems to play a role in how well the components can be resolved. Lower values of $\tau_1$ tend to correspond to better fits, overall. In addition, when looking at two-component spectra, the lifetime values for the second component seem to be somewhat easier to fit. These findings seem to extend throughout the trials.

Of interest is how the relative intensities of the components seem to affect the trials more generally, or at least when looking at the two component trials. Trends in the 20\%-80\% and 50\%-50\% data tend to reverse when looking at the 80\%-20\% data. This effect extends to the second two trials, too. Frustratingly, the source of this reversal remains a mystery, demanding further investigation as to whether this is an error in data processing, or a real effect. Another factor that seems consistent throughout the two component trials is a better fit for $\tau_2$ overall.

Moving on to the second two trials, the findings indicate that while both a decrease in resolution function width and an increase in count number seem to improve the fit, in general, the latter seems to yield more consistent results. The improvement seems to be in both accuracy and precision, and persists even when accounting for the proportional increase in background noise. An interesting direction might be to test what happens if the signal-to-noise ratio is decreased as the count number increases, and to what degree the latter can make up for the former.

The one component fit in the last trial seems to follow the average lifetime. So too does the first component of the two lifetime fit, with the second component attempting to fit the low intensity positronium lifetime. Finally, in the three component fit, of note is that PALSfit seems to consistently underestimate the third lifetime. A general takeaway, which can be applied to any spectrum, is to pay particular attention to the standard deviations. A large standard deviation is obviously undesirable, but a very small one, especially when one defect component is strong, seems to be indicative of an inaccurate fit.

\section{Difficulties}

The biggest challenge over the course of the project seemed to be a lack of tools to quickly generate, fit and compare a large number of spectra. PALSfit admittedly does allow multiple spectra to be linked to a single control file, but sometimes doing so would result in buggy or inconsistent results. As a consequence, the choice was made early on to use separate control files for every spectrum, which made the process of generating and fitting them more laborious. While some development of the aforementioned tools was done over the course of the project, an effort towards further developing these tools to make them widely available and easy to use might be a useful undertaking. 

An example of where the lack of appropriate tools might have been a limiting factor during this project is in the final trial, that didn't make it into the dissertation. The aim of this trial was to use the standard trapping model to simulate a material with an increasing defect concentration, and so a spreadsheet was used to calculate the appropriate lifetime components. Some sort of tool that could import values from this spreadsheet and use them to generate and fit the appropriate spectra would have been quite useful. The fits were generated in the end, but a lack of the time necessary to figure out and develop a way to visualize the data then led to the decision to abandon work on this trial. Here too, access better tools could have sped the process up and made the difference.

\section{Final assessment}

The aim of this project was to fit simulated PALS spectra to gain knowledge that could aid in interpretation of experimental spectra. Ultimately, this was partially successful and some potentially useful information was gained, as illustrated in the findings. More information could potentially be gained from access to more data, with the limiting factor being tools to generate, process and visualize that data. The final conclusion is that there is still work to be done in this space.